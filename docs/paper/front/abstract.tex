\chapter{Abstract}

After carrying out hand surgeries, the patient often has to undergo a lengthy recovery period in order to get hand mobility back to the original, healthy state. This recovery phase is usually accompanied by a dedicated ergo therapist in various therapy sessions, requiring the physical presence of both the patient and the ergo therapist.

In a joint venture of the DHBW Stuttgart and the Katharinenhospital Stuttgart, the possibility of computer aided recovery is explored. The long term goal of the collaboration is for the patient to be able to complete some of the recovery exercises at home, saving time and resources for both the patient and the clinic.
\\\\
This student research project is exploring one particular possibility of achieving this: combining low cost hand tracking devices with the modern web. Hand tracking devices are small hardware devices containing various sensors, capable of producing a virtualized representation of the hand. Hand tracking devices are normally used in the field of \gls{VR} and \gls{AR}, but can arguably also be used to track the post surgery recovery progress. A well known and relatively inexpensive Hand tracking device is the Leap Motion Device Platform. This is the Tracking device primarily used for this project, although the project architecture allows for the possibility to implement support for other tracking devices.

The essence of the project is to gamify the recovery exercises: the patient should be able to play games through a web interface, controlled by Hand Gestures (for example, spreading the thumb to make a spaceship shoot). The Hand Gestures correspond roughly to recovery exercises that would normally have been done together with a therapist. The therapist should be able to configure gestures for a patient that he or she has to get better at in order to aid in recovery. These gestures must then be used by the patient in order to correctly navigate the game. The gameplay should finally be producing monitoring information for the therapist to review, and thus provide evidence for the recovery progress of the patient.

This work presents a possible Architecure and Minimal Viable Product (MVP) implementation for such a system. The core system components are identified and implemented. Furthermore, advise on extending the system and the recommended next steps are given. Finally, concrete Usage Manuals are provided for both the potential end users and future developers.
